\documentclass[main.tex]{subfiles}
\begin{document}

\begin{appendices}
\chapter{CNN Decision Study}
One way to gain a deeper understanding of how the CNN distinguishes between neutrons and gamma-rays is to examine the pulses classified with a high level of confidence. Figure \ref{fig:CNN95} shows examples of pulses classified with more than 95\% confidence. The pulses labeled as neutrons decay more slowly than those labeled as gamma-rays. This is in agreement with expectations as fast neutrons give rise to recoiling protons in the NE213. These recoiling protons tend to activate slow scintillation light components more than electrons do. 

\begin{figure}[h!]
\centering
\begin{subfigure}{.5\textwidth}
  \centering
  \includegraphics[width=\linewidth]{DigitalResults/neutronpulses.pdf}
  \caption{Neutron pulses.}
  \label{fig:CNNneutron}
\end{subfigure}%
\begin{subfigure}{.5\textwidth}
  \centering
  \includegraphics[width=\linewidth]{DigitalResults/gammaRayPulses.pdf}
  \caption{Gamma-ray pulses}
  \label{fig:CNNgamma}
\end{subfigure}
\caption{Pulses classified by the CNN with more than 95\% confidence.}
\label{fig:CNN95}
\end{figure}

Examining the pulses the network had trouble with can further explain the decision making process. The CNN was designed to assign a number between 0 and 1 to pulses, with values close to 0 meaning the network is certain it is looking at a gamma-ray and values close to 1 meaning that it is certain that it is looking at a neutron. 
Figure \ref{fig:occl} shows a pulse with a prediction value of 0.49, meaning that the network has difficulty determining particle species. By forcing a subset of samples in this waveform to zero, the decision of the network is constrained to the remaining information. The yellow curve in Fig. \ref{fig:occl} shows the networks prediction when a 31 pixel window of zeros or ``blanks'' is scanned across the pulse. A prediction at a time $t$ in Fig. \ref{fig:occl} represents the prediction of the network when the 31 samples in the pulse centered around $t$ are set to zero. When blanking the body of the pulse, the prediction rises to around 1, which means the network is convinced it is looking at a neutron, based upon the remaining information in the tail. When the first part of the tail is blanked, the network becomes convinced that it is looking at a gamma-ray, based upon the remaining information. The network has learned that neutrons have a significant fraction of the total charge in the tail of the pulse and that gamma-rays have most of the charge in the body of the pulse.



\begin{figure}[h!]
    \centering
        \includegraphics[width=\textwidth]{DigitalResults/ambiguouspulse.pdf}
        \caption[CNN decision study.]{CNN decision study. The green curve (left y-axis) shows a pulse the network could not clearly classify as either neutron or gamma-ray. The yellow curve (right y-axis) shows how the CNN prediction varies when part of the pulse is blanked.}
    \label{fig:occl} 
\end{figure}


\end{appendices}


\end{document}