\documentclass[main.tex]{subfiles}
\begin{document}
\section{Neutrons and Neutron Detection}
\subsection{Introduction}

\subsection{Neutron Matter Interactions}
The neutron is composed of an up and two down quarks and has no net electric charge. It does however have a small magnetic moment due to the charged quarks it is constituted by. 

Like all particles the neutron is affected by the gravitational force. However, this effect is negligible in most experimental situations.

Although the neutron has no charge its constituent quarks may interact via the electromagnet force. The fact that it has a magnetic moment also means that it can be scattered by magnetic fields.

As a fermion the neutron can also interact via the weak force.

The strong force is responsible for binding quarks together to form neutrons and protons, but the residual strong force is responsible for binding nucleons together. The residual strong force is also responsible for the scattering of nucleons.



\subsection{Sources of Free Neutrons}
Spallation sources
Reactors
Radioactive sources
Actinide/Be

\subsection{Detecting Neutrons and Gamma Rays with Scintillation Detectors}
\textbf{Photo Multiplier Tube}\newline
The signals produced by the neutrons and gamma rays in the scintillators are not in themselves strong enough, so a photomultiplier tube is used to increase the total produced charge. Incident light produces electrons via the photo electric effect at the photo cathode, and these electrons are then accelerated towards an anode by a series of dynodes. Each time an electron reaches a dynode an electron avalanche occurs. The advantage of a PMT is that it makes it possible to amplify small signals by orders of magnitude, and because the amplification is linear information about the original signal strength and timing is preserved.

\subsection{Pulse Shape Discrimination}
\begin{figure}[ht]
	\centering
    	\includegraphics{DigitalSetup/pulse_types.pdf}
        \caption{Typical shapes of neutron and gamma pulses.}
	    \label{fig:pulse_types} 
\end{figure}
\begin{figure}[ht]
    \centering
        \includegraphics{Theory/psd_sketch.pdf}
        \caption{Sketch of a Pulse shape discrimination heatmap. The y-axis denotes charge located in the tail and the x-axis denotes the deposited energy.}
    \label{fig:psd_sketch} 
\end{figure}

\subsection{Neutron Time of Flight and Tagging}
Time of flight as a way to determine energy. can be done with for example a chopper.
Alternative: Tag neutrons.

Neutron tagging
When working with charged particles the energy can for example be calculated from the way in which in which a particles trajectory is bend in a magnetic field. This approach obviously wont work on neutrons, but for fast neutrons the energy can be found by correlating neutrons with gamma photons, which are produced nearly simultaeously. the detection of the neutron gives the end time of its movement from radioactive source to detector, and the gamma ray is used to calculate the start time, and since fast neutrons, E $\approx$ 1-20 MeV, move at classical speeds, we can use classical mechanics to calculate their energy. 
Using a beryllium based source such as Plutonium beryllium we get both neutrons and gammas. The first reaction is as follows: $$^{239}Pu\rightarrow^{234}U+\alpha+\gamma_1$$

The alpha particle can then interact with the beryllium producing both a neutron and another gamma in the following series of events: $$[\alpha+^{9}Be\rightarrow^{12}C^{*}+n],\qquad \longrightarrow \qquad [^{12}C^ {*}\rightarrow^{12}C+\gamma_2]$$

The events all occur on a very short timescale, compared to the time it takes the particles to reach the detectors, so the pairs $\gamma_1-\gamma_2$ and $n-\gamma_1$ can be considered to be produced simultaneuously. %what about n innelastically scattering producing another gamma?

The lab setup is large compared to the source so we assume the particles to come from a single point.
\begin{figure}[ht]
    \centering
        \includegraphics{Theory/tof_sketch.pdf}
        \caption{Sketch of a time of flight spectrum. The arrows indicate neutron and gamma peaks.}
    \label{fig:tof_sketch} 
\end{figure}





\end{document}