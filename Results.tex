\documentclass[main.tex]{subfiles}
\begin{document}

\chapter{Results}

\section{The Data Sets}
The data presented in this chapter was collected during one hour of parallel measurements at the analog and the digital setup. The applied thresholds are shown in table \ref{tab:settings}. The YAP threshold was set at 9.8\si{\milli\volt} during acquisition, but having such a low threshold was found to decrease the signal to noise ratio in the time of flight spectrum, so a higher threshold of 24.4\si{\milli\volt} was applied.
\begin{table}[bh]
\begin{tabular}{|l|l|l|l|l|}
\hline
Setup   & YAP threshold(mV) & NE213 threshold(mV) & NE213 events ($\text{10}^\text{6}$) & Livetime \\ \hline
Analog  & 25.0              & 94.6                & 4.3      & 44\%             \\ \hline
Digital & 9.8/24.4	        & 48.8                & 2.2      & -             \\ \hline
\end{tabular}
\caption{Threshold values and number of NE213 events.}
\label{tab:settings}
\end{table}
\section{The Analog Setup}
\subsection{Pulse Shape Discrimination}
Neutrons and gamma rays were discriminated through the pulse shape parameter: 
\begin{equation}
PS=1-\frac{QDC_{sg}}{QDC_{lg}}
\end{equation}
Where QDC$_\text{lg}$ is the QDC value of the pulse integral over 500 ns and QDC$_\text{sg}$ is the value of the 60 ns integral. The baseline offset will constitute a larger fraction of deposited charge for smaller pulses than for larger pulses, so by adding a constant term to either the longgate or shortgate integration it is possible to change way PS varies as a function of energy. a constant 120 QDC channels was added to the shortgate QDC values, in order to linearize the pulse shape as a function of deposited energy. No constant was added to the longgate QDC values. 

Furthermore, events below 0.8 MeV are mostly the result of the pedestal injection, and consequently they are not of interest for pulse shape discrimination and have been removed. As can be seen from figure \ref{fig:qdc_a} there is some overlap between the injected YAP trigger distribution and the actual PuBe energy spectrum. This will be addressed when we look at the time of flight.

PS is shown as a function of deposited energy in figure \ref{fig:psd_a}. The upper band is made up of pulses for which the tail contained a large fraction of the total charge. This is the neutron band. Conversely The lower band is made up of gamma rays for which most of the energy is carried in the body of the pulse. That this is indeed true will be confirmed by the time of flight information.

\begin{figure}[ht]
    \centering
        \includegraphics{AnalogResults/psd.pdf}
        \caption{Heatmap of the fraction of total integrated charge as a function of energy. The dashed white line indicates the discrimination cut (tail/total = 0.259).}
        \label{fig:psd_a}
\end{figure}
In the analog setup a threshold of 94.6\si{\milli\volt} is applied. Since neutrons have a larger fraction of charge in the tail a 100 mV amplitude neutron signal can be expected to deposit more energy in the detector than a gamma ray pulse of equal amplitude. This gives rise to the curved energy threshold we see in the figure. It is also notable that the neutron band seems to be one single distribution, whereas the gamma band has two clear peaks. This is because the neutrons are produced with a more continuous distribution of energies whereas the gamma rays are produced with specific energies in the deexcitation of nuclei.

The linearization made it possible to draw a straight line through the spectrum, which separates neutrons from gammas. The procedure by which this cut was determined will be presented in parallel for the digital and the analog setup in chapter \ref{sec:results}. For now we just note that PS=0.259, shown as a dashed line in figure \ref{fig:psd_a}, was found to provide the best separation. Seeing how the neutron and gamma distributions seem to overlap at low energies this cut will likely cause some misclassification. Studying the time of flight can help us to get an idea of the extent of this misclassification.


\subsection{Time of Flight spectrum}
By correlating signals in the NE213 detector with those in the YAP we can tag neutrons and gammas in order to construct a time of flight spectrum. For practical reasons NE213 signals are used as start signals for the TDC while the gamma signals are delayed and used as stop signals. Thus the raw time of flight spectrum will contain a neutron peak to the left of a gamma peak. In figure \ref{fig:tof_a} this has been accounted for by switching the sign, converting from TDC channels to nanoseconds using the time calibration factor found in chapter \ref{sec:timecal}, and shifting the gamma peak to time t = 1.055 m/c (the approximate time it takes light to reach the NE213 detector). The energies of particles in the neutron peak is shown in the insert in the figure.

\begin{figure}[ht]
    \centering
        \includegraphics{AnalogResults/tof.pdf}
        \caption{The time calibrated time of flight spectrum. The x-axis denotes the time of flight from source to NE213 detector. The neutron and gamma peaks have been indicated with arrows and the upper right insert shows the energy distribution of events located in the neutron peak}
    \label{fig:tof_a}
\end{figure}

Since all the gamma rays travel at the speed of light one might expect a much narrower gamma peak. The are a number of reasons why this is not the case. First of all the gamma source and the detectors all have some size, so there are multiple paths light can take from source to detector. Secondly interactions are stochastic, so each gamma ray will travel some distance into a detector before interacting, which is particularly relevant for the much larger NE213 detector. The distance travelled through cables and the various electronic components, will also cause some attenuation, which may lead to differences in risetime of low and high amplitude pulses. This will in turn make the constant fraction discriminator less effective causing some loss in the time resolution. Furthermore, the final  digitization by the TDCs may cause some loss of resolution.

Since the distance from source to detector is known we can convert the neutron time of flight into Energy. The insert in figure \ref{fig:tof_a} shows the energy of particles located in the neutron peak, with higher flight times mapping to lower energies. The interval used is highlighted by the red dotted lines.


The time of flight information can offer insight into the energy cut made to separate out the pedestal events as well as the pulse shape discrimination cut. Figure \ref{fig:tof_ps_a} shows pulse shape as a function of time of flight, with neutron and gamma distributions highlighted. It is clear that it is not possible to make a discrimination cut on the pulse shape parameter without significant misclassification. 

Often the start and stop signal will be due to random coincidences. On small timescales of a few hundred nanoseconds these events are expected to form a flat background in the time of flight spectrum, as seen in figure \ref{fig:tof_a} beyond 60 ns. Since the events still represent either neutrons or gamma rays one would expect them to be separated into two bands in figure \ref{fig:tof_ps_a}. One above and one below the cut. However the distributions are not far enough from one another for this to be visible.

Another interesting feature is that there is large amount of gammas at higher PS values. This is because not all the unwanted YAP trigger induced events were removed by the cut at 0.8 $\text{MeV}_\text{ee}$ and as these events use the same signal as start and stop, they will appear to be simultaneous. The reason why they appear to have such a high fraction of tail/total lies in the gate lengths. Since the charge integrals of these events are just integrals of what randomly happened to be in the detector at the time a YAP trigger occured, and since the longgate window is more than eight time as long as the shortgate window it is more likely that there is something to integrate in the tail part of the window.
\begin{figure}[ht]
    \centering
        \includegraphics{AnalogResults/tof_psd.pdf}
        \caption{Time of flight plotted against PS. The dashed white line indicates the discrimination cut at tail/total = 0.259. A logarithmic z-axis is used to highlight the distribution of background events. Pedestal events have been removed with a cut at 0.8 $\text{MeV}_\text{ee}$.}
    \label{fig:tof_ps_a} 
\end{figure}

We can gain some more information on the effect of the injected pedestal events by looking at figure \ref{fig:tof_E_a}. Above 0.8 $\text{MeV}_\text{ee}$ We have a gamma and a neutron distribution as marked by the arrows. It is also clear that the events produced by triggering on YAP signals (less than 0.8 MeV) mostly land near the gamma ray time of flight. This is to be expected since these signals are acting as both start and stop signals. It is also clear that the cut at 0.8 $\text{MeV}_\text{ee}$ will not be able to remove all of these events. Unlike the gamma flash the neutron distribution show some correlation between time of flight and energy deposition. It seems that the faster neutrons are able to deposit more energy than the slower ones.

In figure \ref{fig:tof_Edep_Eneutron_a} the deposited energy is shown as a function of neutron kinetic energy as found from the time of flight spectrum. It can be seen that high energy deposition implies high neutron energy. However, the converse is not necessarily true as neutrons may scatter out of the detector before depositing all their energy. It is interesting that the energy the detector sees in $\text{MeV}_\text{ee}$ appears to be only half of the neutrons kinetic energy in  $MeV$.

\begin{figure}[ht]
    \centering
        \includegraphics{AnalogResults/tof_E.pdf}
        \caption{Time of flight plotted against energy deposition.}
    \label{fig:tof_E_a} 
\end{figure}

\begin{figure}[ht]
    \centering
        \includegraphics{AnalogResults/tof_Edep_Eneutron.pdf}
        \caption{Neutron energy as a function of deposited energy in the NE213 detector.}
    \label{fig:tof_Edep_Eneutron_a} 
\end{figure}



\section{Digitized Waveform Neutron Tagging}

\subsection{Charge Comparisson PSD}
Just like with the analog setup the charge integrals were used to perform pulse shape discrimination. The resulting PSD heatmap is shown figure \ref{fig:psd_d}. Since a lower amplitude threshold was applied to the digital setup we find a lot of low energy gamma rays in the gamma band. The 2.23 $MeV_{ee}$ and 4.44 $MeV_{ee}$ compton edges are also clearly distinguishable. 

As with the analog setup longgate and shortgate offsets can be used to linearize the seperation between the bands. Here each shortgate sample point has been offset by 4.79 mV while each longgate sample has been offset by 0.24 mV. At lower energies the bands intersect, but above 1.5 MeV they are clearly separated.

\begin{figure}[ht]
    \centering
        \includegraphics{DigitalResults/psd.pdf}
        \caption{Pulse shape discrimination spectrum produced with the charge comparisson method.}
        \label{fig:psd_d}
\end{figure}

\subsection{Convolutional Neural Network PSD}
The convolutional neural network described in \ref{sec:cnn} was applied to the digitized waveforms. Since the activation function of the output node is the logistic function the value is bounded between zero and one. The resulting pulse shape discrimination spectrum is shown in figure \ref{fig:cnn_E} as a function of deposited energy. Like for the charge comparisson method the upper distribution is neutrons and the lower distribution consists of gamma rays. The bands are better separated here than by the charge comparison method although there still is some slight overlap at low energies. Again the 2.23 and 4.44 $MeV_{ee}$ Compton edges are clearly visible in the gamma band. 

\begin{figure}[ht]
    \centering
        \includegraphics{DigitalResults/CNNpsd.pdf}
        \caption{Pulse shape discrimination spectrum produced with a convolutional neural network.}
    \label{fig:cnn_E} 
\end{figure}

\clearpage
\subsection{Time of flight spectrum}
The time of flight spectrum acquired in the 1 hour of data taking is shown in figure \ref{fig:tof_d}. The gamma and netron peaks are indicated with arrows. The gamma peak is not entirely narrow as one might expect seeing as they all travel at the speed of light. As for the analog setup part of the reason is that each photon may have a longer or shorter flightpath depending on its point of production in the source and the point at which it interacts in the detector. Furthermore, the time resolution is limited by is limited by the digitization process. The CFD algorithm looks for where the pulse crosses 30\% of maximum amplitude, but the determination of the maximum amplitude is limited by the resolution and sampling rate of the digitizer.

The flight times in the neutron peak, marked with dotted red lines, has been used to generate the energy spectrum shown in the upper right insert of figure \ref{fig:tof_d}. The spectrum goes to lower energies than what we see for the analog setup in figure \ref{fig:tof_a}, this may be because the digital setup has a lower amplitude threshold than the analog setup, allowing slower neutrons to be recorded.
\begin{figure}[ht]
    \centering
        \includegraphics{DigitalResults/tof.pdf}
        \caption{The time of flight spectrum of the PuBe source produced by the digital setup.}
    \label{fig:tof_d} 
\end{figure}

In order to test the two pulse shape discrimination algorithms one can plot their pulse shape parameters against time of flight. This is shown in figure \ref{fig:tof_cc_tof_cnn}. Figure \ref{fig:tof_digi_cc} shows the narrow gamma distribution and the wider neutron distribution as separated by the charge comparisson method. It is apparent that the two distributons overlap somewhat in pulse shape. It is of note that the neutron and gamma background forms two slightly separated bands.  In figure \ref{fig:tof_digi_cnn} it can be seen that the CNN method provides a better separation, although it still appears that gamma and neutron distributions overlap slightly near prediction value 0.5. The distribution of random coincidence events clearly separate into a gamma ray band below the cut and a neutron band above it.


\begin{figure}
    \centering
    \begin{subfigure}[ht]{\textwidth}
        \includegraphics{DigitalResults/tof_psd.pdf}
        \caption{}
        \label{fig:tof_digi_cc}
    \end{subfigure}
	\begin{subfigure}[ht]{\textwidth}
        \includegraphics{DigitalResults/CNNtof_psd.pdf}
        \caption{}
        \label{fig:tof_digi_cnn}
    \end{subfigure}
    \caption{Heatmaps of of pulseshape discrimination parameters as functions of time of flight plotted with logarithmic z-axis.}
    \label{fig:tof_cc_tof_cnn}
\end{figure}

The heatmap shown in figure \ref{fig:tof_E_d} shows energy deposition in the NE213 detector as a function of time of flight. This spectrum hints at why the gamma ray time of flight peak is not entirely narrow. It seems that low energy gamma rays are a bit more spread out regarding time of flight. This could mean that gamma rays traveling at an angle (and thus travelling a slightly longer path) have a higher probability of hitting the detector closer to the edge and get scattered before depositing all of their energy. It could also indicate that the CFD algorithm has a harder time timestamping low charge pulses.

The neutron distribution clearly shows a relation between time of flight and deposited energy. In figure \ref{fig:tof_Edep_Eneutron_d} this relation is highlighted. There appears to be a roughly linear relation between maximum deposited energy and neutron kinetic energy (which determines time of flight). However, as for the analog setup a neutron may deposit all or only part of its energy, so for a given deposited energy we can only conclude the minimum amount of energy the neutron must have had.


\begin{figure}[ht]
    \centering
        \includegraphics{DigitalResults/tof_E.pdf}
        \caption{Time of flight plotted against energy deposition.}
    \label{fig:tof_E_d} 
\end{figure}

\begin{figure}[ht]
    \centering
        \includegraphics{DigitalResults/tof_Edep_Eneutron.pdf}
        \caption{Neutron energy as a function of deposited energy in the NE213 detector.}
    \label{fig:tof_Edep_Eneutron_d} 
\end{figure}







\section{Results and Performance Comparisson}\label{sec:results}
\subsection{Energy Deposition}

The analog and the digital setup were run with different amplitude thresholds, so in order to properly compare them the correct threshold need to be found. Since the two setups have different cable lengths to the detectors they are not attenuated to the same extent, so the digital setup will need a significantly higher threshold than the analog setup in order to accept the same events. Figure \ref{fig:qdc_comp} top panel shows the energy and time of flight spectra. It is clear from the energy spectrum that the digital setup has a lower threshold applied In addition to this the analog setup reaches the limit of the QDCs range slightly above \SI{6}{\MeV}$_\text{ee}$, whereas the digital spectrum continues. 

\begin{figure}[h]
    \centering
        \includegraphics[width=0.8\textwidth]{CompareResults/qdc_comp.pdf}
        \caption{Comparison of analog and digital time of flight spectra. In the lower panel livetime has been taken into account and the digitized data has had a higher threshold enforced on the NE213 signals.}
    \label{fig:qdc_comp}
\end{figure}

In the bottom panel of the figure the threshold has been adjusted to match the analog setup and the counts of the analog setup have been rescaled to compensate for 56\% deadtime. The data acquisition software WaveDump does not offer a way to calculate deadtime\footnote{Although it is possible to make custom software that does this using CAENs libraries}, so instead the digitized data has been scaled to match the data from the analog daq. By carrying out these changes the two energy spectra look more alike, although the \SI{4.44}{\MeV}$_\text{ee}$ Compton edges look quite different. This may be because some of the high energy gamma rays were outside of the dynamic range of the digitizer.

Figure \ref{fig:qdc_ratio} shows the ratio between the digital and analog Energy spectra from after livetime adjustments and threshold alignment. Below \SI{0.8}{\MeV}$_\text{ee}$ the analog setup will pull the ration down due to the pedestal injection and above \SI{6}{\MeV}$_\text{ee}$ the ratio blows up because this is outside the range of the analog QDC modules. Ideally however everything inbetween 0.8 and \SI{6}{\MeV}$_\text{ee}$ should be very close to 1. Initially this seems to be the case, but near the \SI{4.44}{\MeV}$_\text{ee}$ compton edge there is a bump followed by a vally. This is likely because the highest amplitude digitized pulses were clipped, causing them to be pushed to lower values of deposited energy.
\begin{figure}[h]
    \centering
        \includegraphics[width=0.8\textwidth]{CompareResults/QDC_ratio.pdf}
        \caption{Ratio of the digital and analog QDC spectra.}
    \label{fig:qdc_ratio}
\end{figure}

\subsection{Time of Flight Spectra}
Like the energy spectrum the time of flight is heavily influenced by the choice of amplitude threshold. Figure \ref{fig:tof_comp} top panel shows the time of flight spectra for the two setups with the intial 49 mV NE213 threshold on the digital setup and 94.6 mV on the analog setup. Interestingly it seems that the analog setup is cutting away the slower neutrons (this does not mean that they are not \textit{fast} neutrons) by setting the amplitude threshold too high. By applying the new NE213 threshold of 151 mV to the digital setup and adjusting the livetime of both setups in the same manner as for the energy spectrum the time of flight spectrum shown in the bottom panel of figure \ref{fig:tof_comp} is obtained. A couple of things stand out here. First and foremost there are more counts in the digitized time of flight spectrum. This is because the YAP thresholds have not been aligned. Secondly the neutron peaks now have roughly the same shape, which means that the amplitude cut has removed the slower of the fast neutrons. Additionally, with a full width at half maximum of the gamma peak of \SI{7.67}{\ns} the analog setup seems to have a poorer time resolution than the digital setup which has a FWHM of 2.75 ns at the gamma peak.

\begin{figure}[h]
    \centering
        \includegraphics[width=0.8\textwidth]{CompareResults/tof_comp.pdf}
        \caption{Comparison of analog and digital time of flight spectra. In the lower panel livetime has been taken into account and the digitized data has had a higher threshold enforced on the NE213 signals.}
    \label{fig:tof_comp}
\end{figure}

\subsection{PSD Figure of Merit}
By fitting Gaussian functions to the neutron and gamma distribution shown in \ref{fig:fom_analog} and \ref{fig:fom_digital} one can express the quality of separation between the two distribution as a figure of merit, FoM, defined in terms of the centers, C, of the gaussians and their full width half maximum. This way of parametrizing the quality of a PSD method requires that the distributions are approximately Gaussian. This assumption appears to be valid for the neutron distribution, centered at 0.3, but both figure \ref{fig:fom_analog} and figure \ref{fig:fom_digital} shows a tail on the gamma distribution which is not matching with the fit. The assumption of Gaussian distributions is not valid for the distributions produced by the CNN method, so the FoM will only be used for comparing the charge comparison methods. 

In order to fit the Gaussians the raw PS histograms were first smoothened with a Gaussian kernel, which made it possible to locate extreme values. The Gaussians were then fitted. The two distributions are clearly better separated for the digital setup than for the analog setup. 

\begin{equation}
FoM = \frac{C_n - C_\gamma}{FWHM_n + FWHM_\gamma}
\end{equation}

\begin{figure}[ht]
	\begin{subfigure}[b]{\textwidth}
	    \centering
    	    \includegraphics[width=0.8\textwidth]{CompareResults/FOM_analog.pdf}
        	\caption{Analog setup}
	    \label{fig:fom_analog} 
	\end{subfigure}
	\begin{subfigure}[b]{\textwidth}
    	\centering
        	\includegraphics[width=0.8\textwidth]{CompareResults/FOM_digital.pdf}
        	\caption{Digital setup}
    	\label{fig:fom_digital} 
    \end{subfigure}
    \caption{The blue histograms shows the pulse shape parameter. The green curve is this same histogram after being smoothed by a Gaussian kernel, to allow for identification of extreme values. The red and green curves are fits to the gamma ray and neutron distributions respectively. The inserts show the pulse shape parameter as a function of energy, for full size version see fig \ref{fig:psd_a} and \ref{fig:psd_d}. }
\end{figure}

Looking at the inserts in both figures it can be seen that the FoM is highly energy dependent. This energy dependence is illustrated by figure \ref{fig:psd_fom_trend}. The digital setup performs better at all energy thresholds and only at \SI{3}{MeV} the analog reaches the FoM the digital setup had at \SI{0.4}{\MeV}. The ideal placement of the pulse shape discrimination cut also varies with energy, but as is shown in figure \ref{fig:psd_cut_trend} the linearizations of the pulse shape parameters have minimized these variations.
\begin{figure}[ht]
	\begin{subfigure}[b]{\textwidth}
	    \centering
    	\includegraphics[width=0.8\textwidth]{CompareResults/PSD_comp.pdf}
        \caption{}
	    \label{fig:psd_fom_trend} 
	\end{subfigure}
	\begin{subfigure}[b]{\textwidth}
    	\centering
        \includegraphics[width=0.8\textwidth]{CompareResults/PSD_cut.pdf}
        \caption{}
    	\label{fig:psd_cut_trend} 
    \end{subfigure}
    \caption{(a) Pulse shape discrimination figure of merit plotted as a function of minimum deposited energy. (b) Ideal pulse shape cut as a function of minimum deposited energy.}
\end{figure}


\subsection{Misclassification rate}
Another way to compare the performance of the PSD methods is by estimating the misclassification rate. This can be done by evaluating the time of flight spectrum in three different regions and comparing the relative number of neutron and gamma ray labeled events. Ideally the number of gamma rays identified per nanosecond channel in the background region should be the same as the number of gamma rays identified in the neighborhood of the neutron time of flight peak. Likewise the number of neutrons identified at the gamma peak should correspond to the neutron background.

This definition of misclassificaton rate is relying on the assumption that the background is approximately flat and that it has been determined fairly successfully.

Figure \ref{fig:tof_cc_cnn} shows the time of flight spectrum obtained with the analog setup filtered according to the charge comparison method, as well as the spectrum obtained from the digital setup, filtered by charge comparison and by the CNN. Gamma rays are colored blue, neutrons red and their intersection is purple. 

For the analog setup a high degree of contamination is evident. From \ref{fig:tof_ps_a} it was found that a number of the injected YAP start triggers remained even after cutting away events below \SI{0.8}{\MeV}, and that these landed at the gamma ToF but with very high PS values. These events will be part of the reason why there is nearly 12\% misclassification near the gamma peak.

The digital setup provides a lower misclassification rate with the charge comparison method, at \textgamma-n region: 10.96\% error and \textgamma-\textgamma\; region : 3.56\% error. The CNN approach reaches even better results with \textgamma-n region: 6.53\% error and \textgamma-\textgamma\; region : 2.63\% error.

The neutron background is found to be nearly the same by the analog and digital charge comparison methods, at 37.32\% and 38.38\% respectively. The CNN method finds a significantly higher background of neutron events at 44.05\%. 

The correct fraction of events due to neutrons interacting in the NE213 is not trivial to determine. Although the source is approximately isotropic radiation scattered by the aquarium and walls of the room complicates matters. So to get a better idea of what fraction of neutrons to expect simulations are needed. It is however clear that the CNN distribution is better at reproducing a flat background of neutrons at the gamma peak and vice versa, than the digital and analog charge comparison methods, so it seems likely that 44.05\% is a better estimate.

------------------------------------------------

NOTES: If a model is biased towards a certain particle species, then the background will also be incorrect. In this way one can minimize the classification error for one species by having a model biased in favor of this species. This will however cause the classification error for the other species to grow. perhaps it is better to use $\sqrt{error_\gamma^2+error_n^2}$.



\begin{figure}
    \centering
    \begin{subfigure}[bh]{\textwidth}
   	   	\centering
	    \includegraphics{AnalogResults/ToF_filt.pdf}
        \label{fig:ToF_filt_A}
    	\caption{Time of flight spectrum filtered by a linear cut at PS = 0.259.}
    	\label{fig:ToF_filt_A}
   	\end{subfigure}
    \begin{subfigure}[bh]{\textwidth}
   	    \centering
        \includegraphics{DigitalResults/ToF_filt.pdf}
        \caption{Time of flight spectrum filtered by a linear cut at PS=0.222.}
        \label{fig:ToF_filt_D}
    \end{subfigure}
	\begin{subfigure}[bh]{\textwidth}
	    \centering
        \includegraphics{DigitalResults/CNNToF_filt.pdf}
        \caption{Time of flight spectrum filtered by a linear cut at prediction = 0.5.}
        \label{fig:ToF_filt_D_CNN}
    \end{subfigure}
	
	\caption{}
    \label{fig:tof_cc_cnn}
\end{figure}






\end{document}