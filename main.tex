\documentclass{article}
\usepackage[utf8]{inputenc}

\title{Thesis}
\author{Rasmus Kjær Høier}
\date{September 2018}

\usepackage{natbib}
\usepackage{graphicx}
\usepackage{subcaption}

\usepackage{subfiles}



\begin{document}

\maketitle

\tableofcontents

\begin{abstract}
-Set the stage. ESS and need for neutron detection + energy determination. charectarize detectors with known sources. ToF use in other contexts. In medicine for PET scans. In echolocation used by bats and toothed whales.
-Advantages of digital techniques
-Results: we can do the same measurements, but switching detectors and using more channels is easier. Varying parameters is easier (Take one dataset and vary parameters in software).
\end{abstract}
\newpage

\section{Introduction}
\subfile{NeutronsAndNeutronDetection.tex}%concepts
\subfile{ExperimentalSetup.tex}%Equipment
\subfile{NeutronTagging.tex}%Implementation
\subfile{WaveformNeutronTagging.tex}%Implementation
\subfile{Results.tex}%Implementation
\section{Discussion}%Reflection
\section{Summary and Outlook}%Reflection





%``I always thought something was fundamentally wrong with the universe'' \citep{adams1995hitchhiker}

\bibliographystyle{plain}
\bibliography{references}
\end{document}
