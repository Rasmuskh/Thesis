\documentclass[12pt]{article}
\usepackage[utf8]{inputenc}
%bibliography stuff
\usepackage[square,numbers]{natbib}
\bibliographystyle{unsrtnat}
\usepackage[nottoc]{tocbibind}%get references in table of contents
%figure related stuff
\usepackage{graphicx, caption}
\usepackage{subcaption}
%mathmode non italic text using \text
\usepackage{amsmath}
%document layout
\usepackage[a4paper,margin=1in]{geometry}
%links in pdf
\usepackage[hidelinks]{hyperref}
%document management
\usepackage{subfiles}
%linespacing
\usepackage{setspace}
\singlespacing
%SI units
\usepackage{siunitx}
\DeclareSIUnit{\sample}{S} %get sample unit
%greekletters in text
\usepackage{textgreek}
%degree symbol
\usepackage{gensymb}


\title{Tagging and Identification of Neutrons with Digitized Waveforms}
\author{Rasmus Kjær Høier}
\date{May 2019}
\pagenumbering{roman}

\begin{document}
\maketitle

\newpage
\setcounter{tocdepth}{3}
\tableofcontents

\newpage
\begin{abstract}
-Set the stage. ESS and need for neutron detection + energy determination. charectarize detectors with known sources. ToF use in other contexts. In medicine for PET scans. In echolocation used by bats and toothed whales.
-Advantages of digital techniques
-Results: we can do the same measurements, but switching detectors and using more channels is easier. Varying parameters is easier (Take one dataset and vary parameters in software).\cite{Leo}
\end{abstract}

\newpage
\onehalfspacing
\pagenumbering{arabic}
\setcounter{page}{1}
\subfile{NeutronsAndNeutronDetection.tex}%concepts
\clearpage
\subfile{ExperimentalSetup.tex}%Equipment
\clearpage
\section{Results}%Reflection
\subfile{NeutronTagging.tex}%Implementation
\subfile{WaveformNeutronTagging.tex}%Implementation
\subfile{Results.tex}%Implementation
\clearpage
\section{Discussion}%Reflection
\clearpage
\section{Summary and Outlook}%Reflection
\clearpage
\bibliography{references}
\end{document}
