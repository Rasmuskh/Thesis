\documentclass[12pt]{report}
\usepackage[utf8]{inputenc}
%bibliography stuff
\usepackage[square,numbers]{natbib}
\bibliographystyle{unsrtnat}
\usepackage[nottoc]{tocbibind}%get references in table of contents
%figure related stuff
\usepackage{graphicx, caption}
\usepackage{subcaption}
%mathmode non italic text using \text
\usepackage{amsmath}
%document layout
\usepackage[a4paper,left=1in, right=1in, top=1.5in, bottom=1.5in]{geometry}
%links in pdf
\usepackage[hidelinks]{hyperref}
%document management
\usepackage{subfiles}
%linespacing
\usepackage{setspace}
\singlespacing
%Equation alighnment
\usepackage{mathtools}
%SI units
\usepackage{siunitx}
\DeclareSIUnit{\sample}{S} %get sample unit
%greekletters in text
\usepackage{textgreek}
%degree symbol
\usepackage{gensymb}
%\onehalfspacing
%titlepage dependencies
\usepackage{tikz}
%circled number (relies on tikz)
\newcommand*\circled[1]{\tikz[baseline=(char.base)]{
            \node[shape=circle,draw,inner sep=1pt] (char) {#1};}}
\usetikzlibrary{calc}
%appendix
\usepackage[toc,page]{appendix}
%List of abbreviations
\usepackage{acronym}
\usepackage{changepage}
%for inserting blank pages
\usepackage{afterpage}
\newcommand\blankpage{%
    \null
    \thispagestyle{empty}%
    \addtocounter{page}{-1}%
    \newpage}
%extra hline in table
\usepackage{hhline}



\begin{document}
\subfile{titlepage.tex}%concepts
\newpage
\pagenumbering{roman}

\blankpage
\begin{abstract}
The advantages of performing neutron-tagging measurements using a waveform digitizer are explored. An existing analog setup consisting of modular crate electronics at the Source-Testing Facility at the Division of Nuclear Physics in Lund, Sweden has been digitally replicated. Neutrons are detected using an organic liquid-scintillator  detector while the corresponding \SI{4.44}{MeV} gamma-rays are detected using inorganic scintillation crystals. The performance of the digitizer-based setup is compared to that of the modular analog setup in terms of neutron and gamma-ray pulse-shape discrimination and time-of-flight. The results obtained using the digitizer-based approach are superior to those obtained using the modular analog approach in all aspects. The digitizer-based approach is then successfully employed both to distinguish between neutrons and gamma-rays via a convolutional neural network and to relate neutron deposition energy to neutron kinetic energy via time-of-flight.
%-Set the stage. ESS and need for neutron detection + energy determination. charectarize detectors with known sources. ToF use in other contexts. 
%-Results: we can do the same measurements, but switching detectors and using more channels is easier. Varying parameters is easier (Take one dataset and vary parameters in software).
\end{abstract}
\blankpage

\newpage
\setcounter{tocdepth}{3}
\setcounter{secnumdepth}{2}
\tableofcontents
\listoffigures
\listoftables

\subfile{LOA.tex}%concepts



\newpage
\pagenumbering{arabic}
\setcounter{page}{1}
%CHAPTER1 INTRO
\subfile{NeutronsAndNeutronDetection.tex}%concepts
\clearpage
%CHAPTER2 METHOD
\subfile{ExperimentalSetup.tex}%Equipment
\clearpage
%CHAPTER3 RESULTS
\subfile{Results.tex}%Implementation
\clearpage
~
\newpage
\subfile{summary.tex}%Equipment
\clearpage
\bibliography{references}

\end{document}
