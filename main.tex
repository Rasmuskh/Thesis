\documentclass[12pt]{report}
\usepackage[utf8]{inputenc}
%bibliography stuff
\usepackage[square,numbers]{natbib}
\bibliographystyle{unsrtnat}
\usepackage[nottoc]{tocbibind}%get references in table of contents
%figure related stuff
\usepackage{graphicx, caption}
\usepackage{subcaption}
%mathmode non italic text using \text
\usepackage{amsmath}
%document layout
\usepackage[a4paper,left=1in, right=1in, top=1.5in, bottom=1.5in]{geometry}
%links in pdf
\usepackage[hidelinks]{hyperref}
%document management
\usepackage{subfiles}
%linespacing
\usepackage{setspace}
\singlespacing
%Equation alighnment
\usepackage{mathtools}
%SI units
\usepackage{siunitx}
\DeclareSIUnit{\sample}{S} %get sample unit
%greekletters in text
\usepackage{textgreek}
%degree symbol
\usepackage{gensymb}
\onehalfspacing
%titlepage dependencies
\usepackage{tikz}
%circled number (relies on tikz)
\newcommand*\circled[1]{\tikz[baseline=(char.base)]{
            \node[shape=circle,draw,inner sep=1pt] (char) {#1};}}
\usetikzlibrary{calc}
%appendix
\usepackage[toc,page]{appendix}
%List of abbreviations
\usepackage{acronym}
\usepackage{changepage}


\begin{document}
\subfile{titlepage.tex}%concepts
\newpage
\pagenumbering{roman}

\begin{abstract}
The Source-Testing-Facility offers scientific infrastructure for easy and reliable development and commisioning of neutron detectors. Neutrons are provided by actinide/Berylium sources and a neutron time of flight setup gives access to a well defined neutron energy spectrum. The time of flight setup is however an under-exploited resource. This is because it is build out of analog electronics components, which need cumbersome fine tuning in order to perform well with new detectors. This thesis explores the potential of performing ToF measurements with a waveform digitizer by comparing the performance of the digitizer based setup to that of the established modular-crate electronics based setup. Additionally the potential of the digitizer based approach is examined by developing a neural network for pulse shape discrimination. 



%-Set the stage. ESS and need for neutron detection + energy determination. charectarize detectors with known sources. ToF use in other contexts. 
%-Results: we can do the same measurements, but switching detectors and using more channels is easier. Varying parameters is easier (Take one dataset and vary parameters in software).
\end{abstract}
\newpage
\setcounter{tocdepth}{3}
\setcounter{secnumdepth}{2}
\tableofcontents
\listoffigures
\listoftables
\subfile{LOA.tex}%concepts



\newpage
\pagenumbering{arabic}
\setcounter{page}{1}
%CHAPTER1 INTRO
\subfile{NeutronsAndNeutronDetection.tex}%concepts
\clearpage
%CHAPTER2 METHOD
\subfile{ExperimentalSetup.tex}%Equipment
\clearpage
%CHAPTER3 RESULTS
\subfile{Results.tex}%Implementation
\clearpage
\subfile{summary.tex}%Equipment
\clearpage
\subfile{appendix.tex}
\clearpage
\bibliography{references}

\end{document}
